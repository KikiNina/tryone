\chapter{Second-order Turbulence Closure Models}

In section \ref{kap-slope} and \ref{kap-jgr}, the General Ocean Turbulence Model (GOTM) was extensively used to investigate turbulence and vertical mixing over sloping topography. The development of GOTM started in 1992 in Hamburg during the PhD time of Hans Burchard. Innovations and new findings were discussed and the code was extended during the following years, until two new developers joined the GOTM team in 1998. With the objective to provide any interested user a turbulence model that allows to choose between several established turbulence parameterizations and which can be extended with biogeochemical models or included in three dimensional models, the developers published a first stable version of GOTM in 1999. Since then, GOTM is broadly used and still under continuous development \citep[][]{gotm1999}.

GOTM is a one-dimensional water-column model that solves the Reynolds-averaged Navier-Stokes equations using the Boussinesq approximation (density is treated as constant unless multiplied with the gravitational acceleration) and assumes pressure to be hydrostatic. The most important aspect of GOTM is, however, the choice of turbulence closure models. 

Turbulence is a main contributor to mixing processes in the ocean and affects mean flow as well as particle movement and the spreading of tracers. Turbulent processes can be calculated with Direct Numerical Simulations (DNS), which solve the full Navier-Stokes equations, but the necessary resources and the computational time exceeds by far what is applicable in the simulation of realistic scenarios. In Large Eddy Simulations (LES) only large scale turbulence is calculated directly whilst the small scale is parameterized. With the progress in data processing and computers, LES is more and more convenient to use, but up to now still too expensive in terms of computational time for most applications. Therefore, turbulent fluxes have to be derived from quantities that can be calculated more efficiently. An important class of models for this problem are one-point statistical closure models, which solve transport equations for the turbulent fluxes. As this introduces new variables, physically reasonable assumptions must be made to ensure the existence of a unique solution to the set of equations \citep[][]{UmlaufBurchard2005a}. These assumptions are the turbulence closure. The order of the model refers to the order to which turbulent fluxes are calculated from transport equations. A second order model solves transport equations for the turbulent fluxes and assumes physically reasonable approximations for the newly introduced third-moment fluxes. In a third order model, transport equations for these third-moment fluxes are derived and closures are assumed on the next level (ZITATSANDER 1998???). State of the art in ocean circulation models today are second order turbulence closures.

In all second order models that consider stratification and rotation effects, transport equations for the turbulent fluxes of momentum $\langle u_i^\prime u_j^\prime \rangle$ ($u_i$ being the velocity components) and buoyancy $\langle u_i^\prime b^\prime \rangle$ and a transport equation of the buoyancy variance $k_b = \langle b^{\prime \, 2} \rangle \slash 2 $ are derived. As the obtained equations are physically motivated, all terms appearing have a clear interpretation and can be compared to data (e.g. microstructure turbulence, high resolution acoustic velocity measurements) or the results of other models. 

One possibility for a turbulence close in second order models are the so called pressure redistribution models. One new term in the flux of momentum equation is the pressure--strain correlation $\Phi_{ij}$, which determines how energy is distributed among the components $\langle u_i u_j \rangle$. In most pressure redistribution models $\Phi_{ij}$ is calculated from approximations for the tendency of turbulence to be isotropic, straining effects and buoyancy input in varying ratios.



The standard approach, that is also used in GOTM, is to relate the turbulent fluxes to the gradient of the considered quantity and to a turbulent diffusivity (or viscosity, for the flux of momentum). The calculation of these turbulent diffusivities $\nu_t$ includes a length and a velocity scale:
\begin{equation}
 \label{turbulence}
 \nu_t = c_\mu \sqrt{k} L.
\end{equation}
