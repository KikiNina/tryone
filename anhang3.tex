\chapter{Second-order Turbulence Closure Models}

In section \ref{kap-slope} and \ref{kap-jgr}, the General Ocean Turbulence Model 
(GOTM) was extensively used to investigate turbulence and vertical mixing over 
sloping topography. The development of GOTM started in 1992 in Hamburg during 
the PhD time of Hans Burchard. Innovations and new findings were discussed and 
the code was extended during the following years, and two new developers joined 
the GOTM team in 1998. With the objective to provide any interested user a 
turbulence model that allows to choose between several established turbulence 
parameterizations and which can be extended with biogeochemical models or 
included in three dimensional models, the developers published a first stable 
version of GOTM in 1999. Since then, GOTM is broadly used and still under 
continuous development \citep[][]{gotm1999}. In this appendix, a brief 
introduction into the basic principles and assumptions of turbulence modeling in 
oceanographic context is given and the different classes of turbulence models 
are shortly introduced. I will not explain the mathematical derivation of 
equations here, as this can be found in a broad variety of books and is far too 
much to be presented in this context. This chapter rather aims on giving the 
reader a general idea of the evolutions that lead to todays turbulence models 
and will therefore only contain mathematical equations when they support the 
readability.

GOTM is a one-dimensional water-column model that solves the Reynolds-averaged 
Navier-Stokes equations using the Boussinesq approximation (density is treated 
as constant unless multiplied with the gravitational acceleration) and assumes 
pressure to be hydrostatic. The most important aspect of GOTM is, however, the 
choice of turbulence closure models.  

\section{Introduction}\label{introturbm}

Turbulence is a main contributor to mixing processes in the ocean and affects 
mean flow as well as particle movement and the spreading of tracers. Turbulent 
processes can be calculated with Direct Numerical Simulations (DNS), which solve 
the full Navier-Stokes equations, but the necessary resources and the 
computational time exceeds by far what is applicable in the simulation of 
realistic scenarios. In Large Eddy Simulations (LES) only large scale turbulence 
is calculated directly whilst the small scale is parameterized. With the 
progress in data processing and computers, LES is more and more convenient to 
use, but up to now still too expensive in terms of computational time for most 
applications. Therefore, turbulent fluxes have to be derived from quantities 
that can be calculated more efficiently. An important class of models for this 
problem are one-point statistical closure models, which solve transport 
equations for the turbulent fluxes. As this introduces new variables, physically 
reasonable assumptions must be made to ensure the existence of a unique solution 
to the set of equations \citep[][]{UmlaufBurchard2005a}. These assumptions are 
the turbulence closure. The order of the model refers to the order to which 
turbulent fluxes are calculated from transport equations. A second order model 
solves transport equations for the turbulent fluxes and assumes physically 
reasonable approximations for the newly introduced third-moment fluxes. In a 
third order model, transport equations for these third-moment fluxes are derived 
and closures are assumed on the next level \citep[e.g. in][]{sander1998}. State 
of the art in ocean circulation models today are second order turbulence 
closures. 

In all second order models that consider stratification and rotation effects, 
transport equations for the turbulent fluxes of momentum $\langle u_i^\prime 
u_j^\prime \rangle$ ($u_i$ being the velocity components) and buoyancy $\langle 
u_i^\prime b^\prime \rangle$ and a transport equation of the buoyancy variance 
$k_b = \langle b^{\prime \, 2} \rangle \slash 2 $ are derived. As the obtained 
equations are physically motivated, all terms appearing have a clear 
interpretation and can be compared to data (e.g. microstructure turbulence, high 
resolution acoustic velocity measurements) or the results of other models.  

\section{Pressure Redistribution Models}\label{pressresm}

One possibility for a turbulence closure in second order models are the so 
called pressure redistribution models. One new term in the flux of momentum 
$\langle u_i^\prime u_j^\prime \rangle$ transport equation is the 
pressure--strain correlation $\Phi_{ij}$, which determines how energy is 
distributed among the components $\langle u_i u_j \rangle$. In most pressure 
redistribution models $\Phi_{ij}$ is calculated explicitly from approximations 
for the tendency of turbulence to be isotropic, straining effects and buoyancy 
input in varying ratios. For the corresponding term that appears in the 
transport equation for buoyancy fluxes, $\langle u_i^\prime b^\prime \rangle$ , 
the pressure--buoyancy--gradient correlation $\Phi_i^b$, another 
general explicit model exists that relates the tendency for isotropic 
turbulence and the effects of shear and stratification with the use of several 
empirical coefficients. Although these pressure redistribution models are 
relatively simple, they still require too much computational time for the 
application in ocean circulation models. 

\section{Explicit Algebraic Models}

For a operational turbulence parametrization, the model assumptions described 
above are further simplified to obtain fully explicit expressions for the 
turbulent fluxes that can easily be solved. These are the explicit algebraic 
models. Under the assumption that the production of turbulence by buoyancy and 
shear is in equilibrium with dissipation of turbulence by viscous effects, the 
transport equations for the turbulent fluxes greatly simplify and can be solved 
analytically. Assuming further that timescales of horizontal advection and 
diffusion are much larger than the vertical timescales, the so--called boundary 
layer assumption, the (vertical) fluxes of momentum and buoyancy turn out to be 
\begin{align}
 \label{bblassum}
 \langle u^\prime w^\prime \rangle &= - \nu_t \partial_z u \\
 \langle v^\prime w^\prime \rangle &= - \nu_t \partial_z v \\
 \langle w^\prime b^\prime \rangle &= - \nu_t^\prime \partial_z b
\end{align}
with $u,\,v,\,w$ being the three velocity components and $z$ the vertical 
coordinate. $\nu_t$ and $\nu_t^\prime$ are called the turbulent or eddy 
viscosity and diffusivity, respectively. They depend on the turbulent kinetic 
energy $k$ and the turbulent length scale $l$ according to
\begin{equation}
 \label{turbdiff}
 \nu_t = c_\mu \sqrt{k} l, \quad \nu_t^\prime = c_\mu^\prime \sqrt{k} l.
\end{equation}
Here, $c_\mu$ and $c_\mu^\prime$ are non-dimensional stability 
functions \citep[][]{gotm1999}.

Consequently, the unknowns left for the calculation of turbulent viscosity and 
diffusivity are the turbulent kinetic energy $k$, the length scale $l$ and the 
stability functions. They can all be derived independently, as outlined in the 
following sections.

\section{Closure for $k$ }\label{kclosure}

The turbulent kinetic energy is defined as
\begin{equation}
 \label{TKE}
 k = \frac{\langle u_i^\prime u_i^\prime \rangle}{2}
\end{equation}
and the following transport equation for $k$ can be derived from the 
Navier-Stokes equations \citep[][]{Rodi1980}:
\begin{equation}
 \label{transpk}
 \partial_t k - \partial_z (\nu_k \partial_z k) = P + B -\varepsilon.
\end{equation}
Terms on the right hand site are the shear and buoyancy production
\begin{equation}
 \label{PandB}
 P = \nu_t M^2, \quad B=-\nu^\prime_t N^2, 
\end{equation}
depending on the shear and buoyancy frequency
\begin{equation}
 \label{MundN}
 M^2 = (\partial_z u)^2 + (\partial_z v)^2, \quad N^2 = \partial_z b,
\end{equation}
respectively. $\varepsilon$ denotes the turbulent dissipation (see next 
sections). The calculation of the turbulent diffusivity $\nu_k$ depends on the 
choice of the turbulence closure model. Different boundary conditions at bottom 
and surface can be derived from the logarithmic law of the wall and under the 
assumptions that $P = \varepsilon$ \citep[][]{gotm1999}.

Most coastal ocean models either solve a simplified form of this transport 
equation or assume local equilibrium between shear and buoyancy production and 
dissipation of turbulent fluxes ($P+B=\varepsilon$). Under this assumption, an 
explicit formula for $k$ can be derived that includes shear and buoyancy 
frequencies and stability functions values only from the last time step 
\citep[][]{UmlaufBurchard2005a}:
\begin{equation}
 \label{explick}
 k = (c_\mu^0)^{-3} (c_\mu M^2 - c_\mu^\prime N^2) l^2,
\end{equation}
 with $c_\mu^0$ being a constant. To avoid unreasonable negative values for 
$k$, a lower limit $k_{\min}$ is prescribed.

\section{Closure for $l$ or $\varepsilon$}

Whilst the closure for $k$ is rather straight forward, the closure for the 
dissipation rate $\varepsilon$ is highly under discussion. The problem of 
determining $\varepsilon$ is equivalent to finding a dissipation length scale 
$l$, via the cascading relation \citep[][]{UmlaufBurchard2005a} 
\begin{equation}
 \label{cascad}
 \varepsilon \sim k^{3 \slash 2} l^{-1}.
\end{equation}
%Here, $c_\mu^0$ is a parameter derived under the assumption of unstratified 
%flows where shear production and dissipation are in equilibrium. This is 
%consistent with the log-layer assumption, where this equilibrium is present by 
%definition. Relation \eqref{cascad}, however, is 
%which is assumed to be valid in stratified (and rotating) flows.

A simple and very famous model is the \cite{blackadar1962} formula:
\begin{equation}
 \label{blackadar}
 \frac{1}{l} = \left( \frac{1}{(\kappa (d_s + z_0^s) )} + \frac{1}{\kappa (d_b + 
z_0^b)} + \frac{1}{l_a} \right).
\end{equation}
$d_s,\, d_b,\, z_0^s,\, z_0^b$ are the distances from surface and bottom and 
the corresponding roughness lengths, respectively. $l_a$ depends on vertically 
integrated values of $k^{1 \slash 2}$ or is set to a fraction of the water 
depth. This model has some disadvantages: Firstly, it depends on dimensional 
constants and coordinates and hence the reference system, what should be 
avoided in turbulence modeling. Secondly, it fails for example in cases of 
stably stratified flows and is inapplicable if the considered problem includes 
no vertical boundaries. The problems in stably stratified flows can be reduced 
by clipping the length scale at a certain maximum. Unfortunately, model results 
in some cases are very sensitive to a constant included in this clipping. 

Another way to obtain a closure is to derive another transport equation for $l$ 
(or $\varepsilon$). Several different approaches have been established that 
reveal good performance in many applications: \cite{mellor1982} suggested a 
transport equation for the product of $k$ and $l$, \cite{burchard1995} one for 
the dissipation rate. \cite{wilcox1998} formulated a transport equation for 
$\omega = \varepsilon \slash k$, which turned out to be a great amendment 
especially in the modeling of turbulence under breaking waves. To generalize all 
these different models, \cite{UmlaufBurchard2003a} suggested a generic transport 
equation for arbitrary fractions of $k$ and $l$: 
\begin{equation}
 \label{genericeq}
 \partial_t (k^m l^n) - \partial_z \left( \frac{\nu_t}{\sigma_{mn}} 
\partial_z(k^m l^n) \right) = k^{m-1} l^n (c_{mn1}P + c_{mn3} G - c_{mn2} 
\varepsilon ).
\end{equation}
The choice of $m$ and $n$ determines the model, e.g. with $m=3 \slash 2$ and 
$n=-1$, \eqref{genericeq} becomes the $\varepsilon$-equation in 
\cite{burchard1995} (note that $\varepsilon \sim k^{3 \slash 2} l^{-1}$, see 
\eqref{cascad}). $\sigma_{mn}$ is constant, but depends on the chosen model. The 
formulation of this generic transport equation allows more direct comparison of 
different established turbulence models and can furthermore be used to derive 
new turbulence models, provided that the determination of the necessary 
parameters is constrained, such that the model behavior is controlled.

A clipping for $\varepsilon$ can be applied to improve the performance in 
stably stratified flows, derived from the findings of \citep{Galperin1988}:
\begin{equation}
 \label{galperin}
 \varepsilon_{\lim} = (c_\mu^0)^{3 \slash 4} \frac{kN}{\sqrt{2} c_{\lim}}.
\end{equation}

\subsection{Calculation of $c_{mn3}$ }

Under stable stratification, the model parameter $c_{mn3}$ mentioned in the 
last section was found to depend on the steady-state Richardson number 
$Ri_{st}$, which indicates equilibrium $P + B = \varepsilon$ in homogeneous 
turbulence and was experimentally determined to be around 0.25 
\citep[][]{UmlaufBurchard2005a}. $c_{mn3}$ is then calculated via
\begin{equation}
 \label{cmn3}
 c_{mn3} = c_{mn2} - (c_{mn1} - c_{mn2}) \frac{c_\mu(Ri_{st})}{c_\mu^ \prime 
(Ri_{st})} \frac{1}{Ri_{st}}.
\end{equation}
For unstable stratification, $c_{mn3} = 1.0$ is used \citep[][]{Rodi1987}.

\section{Stability functions}

The stability functions $c_\mu$ and $c_\mu^\prime$ are rational functions of 
the non-dimensional shear and buoyancy number, $\alpha_M = \tau^2 M^2$ and 
$\alpha_N = \tau^2 N^2$, where $\tau = k \slash \varepsilon $ is the turbulent 
time scale. Coefficients are calculated from a prescribed set of parameters. 

\section{Parameters for the Turbulence Model}

For the calculations carried out in section 2 and 3, we chose a $k-\varepsilon$ 
style dynamic equation.

The parameters for the calculation of the turbulent kinetic energy described in 
section\ \ref{kclosure} are set as follows: The lower boundary condition for 
the transport equation \eqref{transpk} is a no-flux condition derived from the 
assumption $P=\varepsilon$:
\begin{equation}
 \label{lowerBC}
 \nu_k \partial_z k = 0, \quad z=0.
\end{equation}
The turbulent diffusivity $\nu_k$ involved in \eqref{transpk} is set to $\nu_k 
= \nu_t$ and values for the other parameters are found in Tab.\ \ref{tabellchen}

The $\varepsilon$-equation is obtained from the generic equation in 
\eqref{genericeq} with $m=3\slash 2$ and $n=-1$ and includes a couple of 
parameters
\begin{equation}
 \label{transpeps}
 \partial_t \varepsilon - \partial_z ( \nu_\varepsilon \partial_z \varepsilon) 
= \frac{\varepsilon}{k} (c_{\varepsilon 1}P + c_{\varepsilon 3} B 
-c_{\varepsilon 2} \varepsilon),
\end{equation}
with $\nu_\varepsilon = \nu_t \slash \sigma_\varepsilon$. Values of the 
parameters are summarized in Tab.\ \ref{tabellchen}. At the bottom, again a 
flux boundary condition is imposed:
\begin{equation}
 \label{epsBC}
 \frac{\nu_t}{\sigma_\varepsilon} \partial_{\tilde{z}} \varepsilon = - 
(c_\mu^0)^3  \frac{\nu_t}{\sigma_\varepsilon} \frac{k^{3 \slash 2}}{\kappa 
(\tilde{z} +z_0)},
\end{equation}
with $\tilde{z}$ denoting the distance from the bottom, $z_0$ the bottom 
roughness and $\kappa$ the van Karman constant.

For the calculation of the coefficients for the stability functions and 
$c_\mu^0$, the parameter set found in \cite{Canuto2001}, version A, was used.

\begin{figure}[ht]
\begin{tabular}{cccccc}
  $k_{\min}$ & $\sigma_\varepsilon$ & $c_{\varepsilon 1}$ & 
$c_{\varepsilon 2}$ & $Ri_{st}$ & $c_{\lim}$\\
\hline
$10^{-12}$ & $1.3$ & $1.44$ & $1.92$ & $0.25$ & $0.27$ \\
 \end{tabular}
\caption{Parameters}\label{tabellchen}
\end{figure}