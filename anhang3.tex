\chapter{Second-order Turbulence Closure Models}

In section \ref{kap-slope} and \ref{kap-jgr}, the General Ocean Turbulence Model (GOTM) was extensively used to investigate turbulence and vertical mixing over sloping topography. The development of GOTM started in 1992 in Hamburg during the PhD time of Hans Burchard. Innovations and new findings were discussed and the code was extended during the following years, until two new developers joined the GOTM team in 1998. With the objective to provide any interested user a turbulence model that allows to choose between several established turbulence parameterizations and which can be extended with biogeochemical models or included in three dimensional models, the developers published a first stable version of GOTM in 1999. Since then, GOTM is broadly used and still under continuous development \citep[][]{gotm1999}. In this appendix, a brief introduction into the basic principles and assumptions of turbulence modeling in oceanographic context is given and the different classes of turbulence models are shortly introduced. I will not explain the mathematical derivation of equations here, as this can be found in a broad variety of books and is far too much to be presented in this context. This chapter rather aims on giving the reader a general idea of the evolutions that lead to todays turbulence models and will therefore only contain mathematical equations when they support the readability.

GOTM is a one-dimensional water-column model that solves the Reynolds-averaged Navier-Stokes equations using the Boussinesq approximation (density is treated as constant unless multiplied with the gravitational acceleration) and assumes pressure to be hydrostatic. The most important aspect of GOTM is, however, the choice of turbulence closure models. 

\section{Introduction}

Turbulence is a main contributor to mixing processes in the ocean and affects mean flow as well as particle movement and the spreading of tracers. Turbulent processes can be calculated with Direct Numerical Simulations (DNS), which solve the full Navier-Stokes equations, but the necessary resources and the computational time exceeds by far what is applicable in the simulation of realistic scenarios. In Large Eddy Simulations (LES) only large scale turbulence is calculated directly whilst the small scale is parameterized. With the progress in data processing and computers, LES is more and more convenient to use, but up to now still too expensive in terms of computational time for most applications. Therefore, turbulent fluxes have to be derived from quantities that can be calculated more efficiently. An important class of models for this problem are one-point statistical closure models, which solve transport equations for the turbulent fluxes. As this introduces new variables, physically reasonable assumptions must be made to ensure the existence of a unique solution to the set of equations \citep[][]{UmlaufBurchard2005a}. These assumptions are the turbulence closure. The order of the model refers to the order to which turbulent fluxes are calculated from transport equations. A second order model solves transport equations for the turbulent fluxes and assumes physically reasonable approximations for the newly introduced third-moment fluxes. In a third order model, transport equations for these third-moment fluxes are derived and closures are assumed on the next level \citep[e.g. in][]{sander1998}. State of the art in ocean circulation models today are second order turbulence closures.

In all second order models that consider stratification and rotation effects, transport equations for the turbulent fluxes of momentum $\langle u_i^\prime u_j^\prime \rangle$ ($u_i$ being the velocity components) and buoyancy $\langle u_i^\prime b^\prime \rangle$ and a transport equation of the buoyancy variance $k_b = \langle b^{\prime \, 2} \rangle \slash 2 $ are derived. As the obtained equations are physically motivated, all terms appearing have a clear interpretation and can be compared to data (e.g. microstructure turbulence, high resolution acoustic velocity measurements) or the results of other models. 

\section{Pressure Redistribution Models}

One possibility for a turbulence close in second order models are the so called pressure redistribution models. One new term in the flux of momentum transport equation is the pressure--strain correlation $\Phi_{ij}$, which determines how energy is distributed among the components $\langle u_i u_j \rangle$. In most pressure redistribution models $\Phi_{ij}$ is calculated from approximations for the tendency of turbulence to be isotropic, straining effects and buoyancy input in varying ratios. For the corresponding term that appears in the transport equation for buoyancy fluxes, the pressure--buoyancy--gradient correlation $\Phi_i^b$, another general model exists that relates the tendency for isotropic turbulence and the effects of shear and stratification with the use of several empirical coefficients. Although these pressure redistribution models are relatively simple, they still require too much computational time for the application in ocean circulation models.

\section{Explicit Algebraic Models}

For a operational turbulence parametrization, the model assumptions described above are further simplified to obtain fully explicit expressions for the turbulent fluxes that can easily be solved. These are the explicit algebraic models. Under the assumption that the production of turbulence by buoyancy and shear is in equilibrium with dissipation of turbulence by viscous effects, the transport equations for the turbulent fluxes greatly simplify and can be solved analytically. Assuming further that timescales of horizontal advection and diffusion are much larger than the vertical timescales, the so--called boundary layer assumption, the (vertical) fluxes of momentum and buoyancy turn out to be
\begin{align}
 \label{bblassum}
 \langle u^\prime w^\prime \rangle &= - \nu_t \partial_z u \\
 \langle v^\prime w^\prime \rangle &= - \nu_t \partial_z v \\
 \langle w^\prime b^\prime \rangle &= - \nu_t^\prime \partial_z b
\end{align}
with $u,\,v,\,w$ being the three velocity components and $z$ the vertical coordinate. $\nu_t$ and $\nu_t^\prime$ are called the turbulent or eddy viscosity and diffusivity, respectively. They are polynomial functions, which depend on a large amount of derived parameters (for their part again dependent on different empirical parameters differing from model to model) which put shear and buoyancy effects in relation. The only unknowns left for the calculation of turbulent viscosity and diffusivity are the turbulent kinetic energy $k$ and the dissipation rate $\epsilon$. 

\section{Closure for $k$ }

The turbulent kinetic energy is defined as
\begin{equation}
 \label{TKE}
 k = \frac{\langle u_i^\prime u_i^\prime \rangle}{2}
\end{equation}
and a transport equation for $k$ can be derived from the transport equation for the turbulent fluxes of momentum. Most coastal ocean models either solve a simplified form of this transport equation or assume full equilibrium between shear and buoyancy production and dissipation of turbulent fluxes. Under this assumption, an explicit formula for $k$ can be derived that includes shear and buoyancy related numbers only from the last time step. 

\section{Closure for $\epsilon$ or $l$ }

Whilst the closure for $k$ is rather straight forward, the closure for the dissipation rate $\epsilon$ is highly under discussion. The problem of determining $\epsilon$ is equivalent to finding a dissipation length scale $l$, via the cascading relation \citep[][]{UmlaufBurchard2005a}
\begin{equation}
 \label{cascad}
 \epsilon \sim k^{3 \slash 2} l^{-1}.
\end{equation}
%Here, $c_\mu^0$ is a parameter derived under the assumption of unstratified flows where shear production and dissipation are in equilibrium. This is consistent with the log-layer assumption, where this equilibrium is present by definition. Relation \eqref{cascad}, however, is 
%which is assumed to be valid in stratified (and rotating) flows.

A simple and very famous model is the \cite{blackadar1962} formula:
\begin{equation}
 \label{blackadar}
 \frac{1}{l} = \left( \frac{1}{(\kappa (d_s + z_0^s) )} + \frac{1}{\kappa (d_b + z_0^b)} + \frac{1}{l_a} \right).
\end{equation}
$d_s,\, d_b,\, z_0^s,\, z_0^b$ are the distances from surface and bottom and the corresponding roughness lengths, respectively. $l_a$ depends on vertically integrated values of $k^{1 \slash 2}$ or is set to a fraction of the water depth. This model has some disadvantages: Firstly, it depends on dimensional constants and coordinates and hence the reference system, what should be avoided in turbulence modeling. Secondly, it fails for example in cases of stably stratified flows and is inapplicable if the considered problem includes no vertical boundaries. The problems in stably stratified flows can be reduced by clipping the length scale at a certain maximum. Unfortunately, model results in some cases are very sensitive to a constant included in this clipping.

Another way to obtain a closure is to derive a transport equation for $l$. Several different approaches that reveal good performance in many applications been established.




