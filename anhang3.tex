\chapter{Second-order Turbulence Closure Models}

In section \ref{kap-slope} and \ref{kap-jgr}, the General Ocean Turbulence Model 
(GOTM) was extensively used to investigate turbulence and vertical mixing over 
sloping topography. GOTM is a one-dimensional water-column model that solves 
the Reynolds-averaged Navier-Stokes equations using the Boussinesq approximation 
(density is treated as constant unless multiplied with the gravitational 
acceleration) and assumes pressure to be hydrostatic. The most important aspect 
of GOTM is, however, the choice of turbulence closure models. 

In this appendix, a brief 
introduction into the basic principles and assumptions of turbulence modeling 
in 
oceanographic context is given. I will not explain the mathematical derivation 
of equations here, as this can be found in a broad variety of books and is 
beyond the scope of this thesis. This chapter rather aims at providing 
the reader with a general idea of the principles and the nomenclature of todays 
turbulence models and will therefore only contain mathematical equations when 
they support the readability.

At the end of this chapter, the turbulence model and the set of parameters used 
in sections 2 and 3 is outlined.  

\section{Introduction}\label{introturbm}

Turbulence is a main contributor to mixing processes in the ocean and affects 
mean flow as well as particle movement and the spreading of tracers. Therefore, 
turbulence needs to be included in numerical ocean circulation models to some 
extent. Turbulent processes can be calculated with Direct Numerical Simulations 
(DNS), which solve the full Navier-Stokes equations, but the necessary resources 
and the computational costs exceed by far what is applicable in the simulation 
of 
realistic scenarios. For faster calculations, in Large Eddy Simulations (LES) 
only large-scale turbulence is resolved directly whilst the smaller scales are 
parameterized. With the progress in data processing and computers, LES has 
become more 
and more convenient to use, but is up to now still too expensive in terms of 
computational time for most 
applications. Therefore, turbulent fluxes have to be derived from quantities 
that can be calculated more efficiently. An important class of models for this 
problem are (second moments) one-point statistical closure models, which solve 
transport equations for the turbulent fluxes. As these transport equations  
introduce additional unknowns, physically reasonable closure assumptions must 
be made 
to ensure the existence of an accurate solution 
to the set of equations \citep[][]{UmlaufBurchard2005a}. A second order model 
solves transport equations for the second moments of the turbulent fluctuations 
and assumes physically 
reasonable approximations for the higher-order statistical moments appearing in 
these equations. In a 
third order model, transport equations for these third-moment fluxes are derived 
and closures are assumed on the next level \citep[e.g. in][]{sander1998}. State 
of the art in ocean circulation models today are second order turbulence 
closures. 

In all second order models that consider stratification and rotation effects, 
transport equations for the turbulent fluxes of momentum $\langle u_i^\prime 
u_j^\prime \rangle$ and buoyancy $\langle 
u_i^\prime b^\prime \rangle$ and a transport equation of the buoyancy variance 
$k_b = \langle b^{\prime \, 2} \rangle \slash 2 $ are derived. Here, $u_i$ 
are the velocity components, $b$ denotes the buoyancy and $\langle . \rangle$ 
the Reynolds average. The prime $.^\prime$ indicates the deviation from the 
Reynolds average, similar to the notation in section 2 and 3. As the obtained 
equations are directly derived from the Navier-Stokes equations, all terms 
appearing have a clear 
interpretation and can be compared to data (e.g. microstructure turbulence, high 
resolution acoustic velocity measurements) or the results of other models.  

\section{Pressure Redistribution Models}\label{pressresm}

One new term in the flux of momentum 
$\langle u_i^\prime u_j^\prime \rangle$ transport equation is the 
pressure--strain correlation $\Phi_{ij}$, which determines how energy is 
distributed among the components $\langle u_i u_j \rangle$. The closure for 
this quantity is the most crucial assumption that has to be made in turbulence 
models. General and commonly used models for the explicit 
calculation of $\Phi_{ij}$ are pressure redistribution models. These models are 
based on the tendency of unforced turbulence to be isotropic, whereas straining 
effects and buoyancy tend to induce anisotropy. Most of the 
pressure-redistribution models only 
differ in the choice of parameters in this equation. For the corresponding term 
that appears in the 
transport equation for buoyancy fluxes, $\langle u_i^\prime b^\prime \rangle$ , 
the pressure--buoyancy--gradient correlation $\Phi_i^b$, other explicit models 
exists that relates the tendency for isotropic 
turbulence and the effects of shear and stratification with the use of several 
empirical coefficients. TALK TO LARS ABOUT: Although these pressure 
redistribution models are 
relatively simple, they still require too much computational time for the 
application in ocean circulation models. 

\section{Explicit Algebraic Models}

For a operational turbulence parametrization, the model assumptions described 
above have to be further simplified to obtain fully explicit expressions 
for the turbulent fluxes that can easily be solved. An explicit equation in 
this context means that only quantities calculated in the last computational 
time step are used for calculations. These are the explicit algebraic 
models. Under the assumption that the production of turbulence by buoyancy and 
shear is in equilibrium with dissipation of turbulence by viscous effects, the 
transport equations for the turbulent fluxes greatly simplify and can be solved 
analytically. Assuming further that timescales of horizontal advection and 
diffusion are much larger than the vertical timescales, the so--called boundary 
layer assumption, the (vertical) fluxes of momentum and buoyancy turn out to be 
\begin{align}
 \label{bblassum}
 \langle u^\prime w^\prime \rangle &= - \nu_t \partial_z u \\
 \langle v^\prime w^\prime \rangle &= - \nu_t \partial_z v \\
 \langle w^\prime b^\prime \rangle &= - \nu_t^\prime \partial_z b
\end{align}
with $u,\,v,\,w$ being the three velocity components and $z$ the vertical 
coordinate. $\nu_t$ and $\nu_t^\prime$ are called the turbulent or eddy 
viscosity and diffusivity, respectively. They depend on the turbulent kinetic 
energy $k$ and the turbulent length scale $l$ according to
\begin{equation}
 \label{turbdiff}
 \nu_t = c_\mu \sqrt{k} l, \quad \nu_t^\prime = c_\mu^\prime \sqrt{k} l.
\end{equation}
Here, $c_\mu$ and $c_\mu^\prime$ are non-dimensional stability 
functions \citep[][]{gotm1999}. These stability function contains the essence 
of the closure model (and therefore the model parameters introduced with these 
models).

Consequently, calculation of turbulent fluxes only depends on the gradient of 
the transported quantity itself and $\nu_t$ and $\nu_t^\prime$. The the 
unknowns left for the calculation of turbulent viscosity and 
diffusivity are the turbulent kinetic energy $k$, the length scale $l$ and the 
stability functions. They can all be derived independently, as outlined in the 
following sections.

\section{Closure for $k$ }\label{kclosure}

The turbulent kinetic energy is defined as
\begin{equation}
 \label{TKE}
 k = \frac{\langle u_i^\prime u_i^\prime \rangle}{2}
\end{equation}
and the following transport equation for $k$ can be derived from the 
Navier-Stokes equations \citep[][]{Rodi1993} WHICH MODELLING ASSUMPTIONS?:
\begin{equation}
 \label{transpk}
 \partial_t k - \partial_z (\nu_k \partial_z k) = P + B -\varepsilon.
\end{equation}
Terms on the right hand site are the shear and buoyancy production
\begin{equation}
 \label{PandB}
 P = \nu_t M^2, \quad B=-\nu^\prime_t N^2, 
\end{equation}
depending on the shear and buoyancy frequency
\begin{equation}
 \label{MundN}
 M^2 = (\partial_z u)^2 + (\partial_z v)^2, \quad N^2 = \partial_z b,
\end{equation}
respectively. $\varepsilon$ denotes the turbulent dissipation (see next 
sections). The calculation of the turbulent diffusivity $\nu_k$ depends on the 
choice of the turbulence closure model. Different boundary conditions for 
\eqref{transpk} at the bottom 
and surface can be derived from the logarithmic law of the wall and under the 
assumptions that $P = \varepsilon$ \citep[][]{gotm1999}.

Most coastal ocean models either solve a simplified form of this transport 
equation or assume local equilibrium between shear and buoyancy production and 
dissipation of turbulent fluxes ($P+B=\varepsilon$). Under this assumption, an 
explicit formula for $k$ can be derived that includes shear and buoyancy 
frequencies and stability functions values only from the last time step 
\citep[][]{UmlaufBurchard2005a}:
\begin{equation}
 \label{explick}
 k = (c_\mu^0)^{-3} (c_\mu M^2 - c_\mu^\prime N^2) l^2,
\end{equation}
 with $c_\mu^0$ being a constant. To avoid unreasonable negative values for 
$k$, a lower limit $k_{\min}$ is prescribed.

\section{Closure for $l$ or $\varepsilon$}

Whilst the closure for $k$ is rather straight forward, the closure for the 
length scale $l$ is highly under discussion. The problem of 
determining $l$ is equivalent to determining the turbulent dissipation rate 
$\varepsilon$, via the cascading relation \citep[][]{UmlaufBurchard2005a} 
\begin{equation}
 \label{cascad}
 \varepsilon \sim k^{3 \slash 2} l^{-1}.
\end{equation}
%Here, $c_\mu^0$ is a parameter derived under the assumption of unstratified 
%flows where shear production and dissipation are in equilibrium. This is 
%consistent with the log-layer assumption, where this equilibrium is present by 
%definition. Relation \eqref{cascad}, however, is 
%which is assumed to be valid in stratified (and rotating) flows.

A simple and very famous model is the \cite{blackadar1962} formula:
\begin{equation}
 \label{blackadar}
 \frac{1}{l} = \left( \frac{1}{(\kappa (d_s + z_0^s) )} + \frac{1}{\kappa (d_b + 
z_0^b)} + \frac{1}{l_a} \right).
\end{equation}
$d_s,\, d_b,\, z_0^s,\, z_0^b$ are the distances from surface and bottom and 
the corresponding roughness lengths, respectively. $l_a$ depends on vertically 
integrated values of $k^{1 \slash 2}$ or is set to a fraction of the water 
depth. This model has some disadvantages: Firstly, it depends on dimensional 
constants and coordinates and hence the reference system, what should be 
avoided in turbulence modeling. Secondly, it fails for example in cases of 
stably stratified flows and is inapplicable if the problem considered includes 
no boundaries at surface or bottom. The problems in stably stratified flows can 
be reduced 
by clipping the length scale at a certain maximum. Unfortunately, model results 
in some cases are very sensitive to a constant included in this clipping 
\citep[][]{UmlaufBurchard2005a}. 

Another way to obtain a closure is to derive another transport equation for $l$ 
(or $\varepsilon$). Several different approaches have been established that 
showed good performance in many applications: \cite{mellor1982} suggested a 
transport equation for the product of $k$ and $l$, \cite{burchard1995} one for 
the dissipation rate. \cite{wilcox1998} formulated a transport equation for 
$\omega = \varepsilon \slash k$, which turned out to be a great amendment 
especially in the modeling of turbulence under breaking waves. To generalize all 
these different models, \cite{UmlaufBurchard2003a} suggested a generic transport 
equation for arbitrary fractions of $k$ and $l$: 
\begin{equation}
 \label{genericeq}
 \partial_t (k^m l^n) - \partial_z \left( \frac{\nu_t}{\sigma_{mn}} 
\partial_z(k^m l^n) \right) = k^{m-1} l^n (c_{mn1}P + c_{mn3} B - c_{mn2} 
\varepsilon ).
\end{equation}
The choice of $m$ and $n$ determines the model, e.g. with $m=3 \slash 2$ and 
$n=-1$, \eqref{genericeq} becomes the $\varepsilon$-equation in 
\cite{burchard1995} (note that $\varepsilon \sim k^{3 \slash 2} l^{-1}$, see 
\eqref{cascad}). $\sigma_{mn}$ is constant, but depends on the chosen model. The 
formulation of this generic transport equation allows more direct comparison of 
different established turbulence models and can furthermore be used to derive 
new turbulence models, provided that the determination of the necessary 
parameters is constrained, such that the model behavior is controlled.

A clipping for $\varepsilon$ can be applied to improve the performance in 
stably stratified flows, derived from the findings of \citep{Galperin1988}:
\begin{equation}
 \label{galperin}
 \varepsilon_{\lim} = (c_\mu^0)^{3 \slash 4} \frac{kN}{\sqrt{2} c_{\lim}}.
\end{equation}

\subsection{Calculation of $c_{mn3}$ }

Under stable stratification, the model parameter $c_{mn3}$ mentioned in the 
last section was found to depend on the steady-state Richardson number 
$Ri_{st}$. The turbulent Richardson number is defined as $Ri = N^2 \slash 
M^2$, and the steady-state Richardson number indicates equilibrium $P + 
B = \varepsilon$ in homogeneous 
turbulence and was experimentally determined to be around 0.25 
\citep[][]{UmlaufBurchard2005a}. $c_{mn3}$ is then calculated via
\begin{equation}
 \label{cmn3}
 c_{mn3} = c_{mn2} - (c_{mn1} - c_{mn2}) \frac{c_\mu(Ri_{st})}{c_\mu^ \prime 
(Ri_{st})} \frac{1}{Ri_{st}}.
\end{equation}
For unstable stratification, $c_{mn3} = 1.0$ is used \citep[][]{Rodi1987}. 
\cite{UmlaufBurchard2005a} showed that $Ri_{st}$, and thus $c_{mn3}$ determines 
the entrainment velocity in stratified boundary layers, which is essential for 
the process investigated in chapters \ref{kap-slope} and \ref{kap-jgr}.

\section{Stability functions}

The stability functions follow directly from the (closed) second-moment 
equations. Depending on the assumptions in the turbulence closure, the 
stability functions $c_\mu$ and $c_\mu^\prime$ include 
constant formulations or simple dependences on the Richardson number for 
highly simplified turbulence closures. More
general assumptions in the turbulence closure lead to rational stability 
functions depending on the non-dimensional shear and buoyancy number, $\alpha_M 
= \tau^2 M^2$ and $\alpha_N = \tau^2 N^2$, where $\tau = k \slash \varepsilon $ 
is the turbulent 
time scale. Coefficients are calculated from the prescribed set of parameters 
that determine the pressure--redistribution model.


Please note that the Richardson number in the last section is $Ri = N^2 \slash 
M^2$, and the stability functions therefore can evaluated for $Ri_{st}$.

\section{Parameters for the Turbulence Model}

For the calculations carried out in section 2 and 3, we chose a $k-\varepsilon$ 
style dynamic equation.

The lower boundary condition for 
the transport equation \eqref{transpk} is a no-flux condition derived from the 
assumption $P=\varepsilon$:
\begin{equation}
 \label{lowerBC}
 \nu_k \partial_z k = 0, \quad z=0.
\end{equation}
The turbulent diffusivity $\nu_k$ involved in \eqref{transpk} is set to $\nu_k 
= \nu_t$ and values for the other parameters are found in Tab.\ \ref{tabellchen}

The $\varepsilon$-equation is obtained from the generic equation in 
\eqref{genericeq} with $m=3\slash 2$ and $n=-1$ 
\begin{equation}
 \label{transpeps}
 \partial_t \varepsilon - \partial_z ( \nu_\varepsilon \partial_z \varepsilon) 
= \frac{\varepsilon}{k} (c_{\varepsilon 1}P + c_{\varepsilon 3} B 
-c_{\varepsilon 2} \varepsilon),
\end{equation}
and with $\nu_\varepsilon = \nu_t \slash \sigma_\varepsilon$. Values of the 
parameters are summarized in Tab.\ \ref{tabellchen}. At the bottom, again a 
flux boundary condition is imposed:
\begin{equation}
 \label{epsBC}
 \frac{\nu_t}{\sigma_\varepsilon} \partial_{\tilde{z}} \varepsilon = - 
(c_\mu^0)^3  \frac{\nu_t}{\sigma_\varepsilon} \frac{k^{3 \slash 2}}{\kappa 
(\tilde{z} +z_0)},
\end{equation}
with $\tilde{z}$ denoting the distance from the bottom, $z_0$ the bottom 
roughness and $\kappa$ the van Karman constant.

For the calculation of the coefficients for the stability functions and 
$c_\mu^0$, the parameter set found in \cite{Canuto2001}, version A, was used.

\begin{figure}[ht]
\begin{tabular}{cccccc}
  $k_{\min}$ & $\sigma_\varepsilon$ & $c_{\varepsilon 1}$ & 
$c_{\varepsilon 2}$ & $Ri_{st}$ & $c_{\lim}$\\
\hline
$10^{-12}$ & $1.3$ & $1.44$ & $1.92$ & $0.25$ & $0.27$ \\
 \end{tabular}
\caption{Parameters for the turbulence model used in section 
2 and 3.}\label{tabellchen}
\end{figure}