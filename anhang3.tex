\chapter{Second-order Turbulence Closure Models}

In section \ref{kap-slope} and section \ref{kap-jgr}, the General Ocean Turbulence Model (GOTM) was extensively used to investigate turbulence and vertical mixing over sloping topography. The development of GOTM started in 1992 in Hamburg during the PhD time of Hans Burchhard. Innovations and new findings were discussed and the code was extended during the following years, until two new developers joined the GOTM team in 1998. With the objective to provide any interested user a turbulence model that allows to choose between several established turbulence parametrisations to set up customized testcases and which can be extended with biogeochemical models or included in three dimensional models, the developers published a first stable version of GOTM in 1999. Since then, GOTM is broadly used and still developing further \citep[][]{gotm1999}.

GOTM is a one-dimensional water-column model that solves the Reynolds-averaged Navier-Stokes equations using the Boussinesq approximation (density is treated as constant unless multiplied with the gravitational acceleration) and assumes pressure to be hydrostatic. The most important aspect of GOTM is, however, the choice of turbulence closure models. 

Turbulence is a main contributor to mixing processes in the ocean and affects mean flow as well as particle movement and the spreading of tracers. Turbulent processes can be calculated with Direct Numerical Simulations (DNS), which solve the full Navier-Stokes equations, but the necessary resources and the computational time exceeds by far what is applicable in the simulation of realistic scenarios. In Large Eddy Simulations (LES) only large scale turbulence is calculated directly whilst the small scale is parameterized. With the progress in data processing and computers, LES is more and more convenient to use, but up to now still too expensive in terms of computational power for most applications. Therefore, turbulent fluxes have to be derived from quantities that can be calculated more efficiently \citep[][]{UmlaufBurchard2005a}. The standard approach, that is also used in GOTM, is to relate the turbulent fluxes to the gradient of the considered quantity on the one hand, and to a turbulent diffusivity (or viscosity, for the flux of momentum) on the other hand. The calculation of these turbulent diffusivities $\nu_t$ includes a length and a velocity scale:
\begin{equation}
 \label{turbulence}
 \nu_t = c_\mu \sqrt{k} L.
\end{equation}
