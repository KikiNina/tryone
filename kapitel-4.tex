\chapter{Sediment transport over sloping topography}
\label{kap-slope}

\section{Motivation}

To investigate basin scale sediment transport pathways, a distinct knowledge about the underlying, small scale processes is crucial. Motivated by the outcomes of studies on estuarine dynamics, which trigger sediment transport processes, a basic process study was carried out. The impact of an oscillatory current moving over sloping topography with overlying stratified water body on suspended matter transport was examined. 

The basin structure and the permanent salinity stratification provide the requirements for the above process in large areas of the Baltic Sea. Although tides do not play a significant role, other oscillatory currents are found in the Baltic: near-inertial waves, internal seiching motions in the basins, topographic \citep[][]{holtermann2012} or coastal-trapped waves \citep[][]{pizarro1998}.

In the absence of tides and due to the permanent deep-water stratification, inertial oscillations and low-mode near-inertial wave motions have been found to dominate the energetics on the basin-scale \citep[][]{vanderlee2011}. Near-inertial internal waves have periods close to the inertial period, which is $T_f = 14.8$ h in the southern part of the Baltic Sea (at a latitude of $54^\deg$N) down to $T_f = 13.1h$ in the northern part ($66^\deg$N). 

Tide gauge records along the coast indicate that eigenoscillations play a major role in the Baltic Sea \citep[][]{wubber1979}. Large-scale meteorological forcing 