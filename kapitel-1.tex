\chapter{Introduction}
\label{kap-einleitung}

% Seitennummerierung auf arabische Zahlen umstellen, neu beginnend im Kapitel 1
\pagenumbering{arabic}

%\chapterquote{Dämliches Zitat}{Zitierter Mensch}

\section{The Baltic Sea}

\subsection{History and Present State}

The Baltic Sea emerged from a subglacial lake at the end of the Weichselian Ice Age, approximately 12,000 BP, and has since gone through several stages of salty and brackish sea states or fresh water conditions, weather a connection to the open ocean was established or not. Indicators for this alternations are, amongst others, changes in the benthic communities, and so the nomenclature of the evolution stages was often determined by the prevailing species.

With the retreat of the glaciers at the end of the Weichselian Ice Age, a subglacial lake, together with water from the rapidly melting glaciers, formed the \textit{Baltic Ice Lake}. Land uplift, that followed the deglaciation, formed a landbridge between Sweden and the main land 11,200 BP, damming the Baltic Ice Lake. Sea level was raised by glacial melt water until the lake began to drain over the Öresund sill. Presumably, a lowering of the water level by 25 to 28 m by drainage through a channel near Mt. Billingen in central Sweden happened very rapidly around 10,300 BP, when the retreating ice margin collapsed \citep[][]{bjoerk95,tikkanen2002} and a connection to the sea was opened. 

This new stage of the Baltic with a lower sea level is known as \textit{Yoldia Sea} (from the marine mollusc \textit{Portlandia (Yolida) arctica} \citep[][]{schoning2001}). There are no indicators for salt water entrainment into the Yoldia Sea until global sea level rise and deglaciation formed a connection through the Närke Strait 300 years later \citep[][]{schoning2001}. This lead to brackish water conditions followed by decreasing salinity after the straits had become too narrow to allow water exchange another 100 to 300 years later \citep[][]{bjoerk95}.

After the glaciers retreated, uplift of the land was not be compensated by global sea level rise. Connection to the ocean was closed again around 9,500 BP, and the regime shifted to fresh water conditions again, now called \textit{Ancylus Lake} (after the freshwater gastropod \textit{Ancylus fluviatilis} \citep[][]{tikkanen2002}). Although narrow outlets existed, the surface of the Ancylus Lake rose up to 10 m above global sea level, inundating extensive areas of land. This phase, called the Ancylus Transgression, ended around 9,200 BP, when the sea level height exceeded the height of the Darss Sill and water began to flow out through an evolving river system in the land connection between Sweden and Germany \citep[][]{tikkanen2002}. Around 9,000 BP this so-called Ancylus Regression ended when the water level reached global sea level. The connection ti the ocean through the river system remained, but no extensive saline intrusion happened for the next 1,000 years.

Ongoing global sea level rise deepened the connection between the Baltic Basin and the open ocean, letting more and more saline water enter the Ancylus Lake. A short transition stage called \textit{Mastogloia Sea} (\textit{Mastogloia} are diatoms that live in slightly saline conditions \citep[][]{eronen2001}) was followed by the \textit{Litorina Sea} (named after the brackish water gastropod \textit{Littorina littorea} \citep[][]{eronen2001}), when a great extent of saline water entered the Baltic basin through the Straits of Denmark in 7,500 BP \citep[][]{bjoerk95}. Transgression took place until the rise in ocean levels ended 6,000 to 5,000 BP. When the Straits of Denmark became shallower, as land uplift continued, water exchange was reduced and salinity in the Litorina Sea declined down to the present brackish state in 4,000 BP, at first named \textit{Limnea Sea} (from the snail \textit{Lymnaea ovata}), and finally, since 500 BP, \textit{Mya Sea} (after the Northern American clam \textit{Mya arenaria}, which was brought into the Baltic by Viking ships \citep[][]{bjoerck2008}).

Today, the Baltic Sea has an extent of $4.2 \times 10^{5} \, \text{km}^2$ \citep[][]{balticsea} and is connected to the North Sea via the Danish Straits: the \O resund and the Great and Little Belt together with the Fehmarn Belt. The climate is humid: Annual precipitation adds around 224 $\text{km}^3$ of water to the Baltic Sea, evaporation amounts to only 184 $\text{km}^3$. River discharge a total of 436 $\text{km}^3$ water per year. \citep[][]{reissmann2009}. An exchange flow takes place, where brackish water leaves the Baltic near the surface (annually around 947 $\text{km}^3$) and saline North Sea water enters at the bottom (nearly 500 $\text{km}^3$ a year on average, but frequency and strength of inflow events differ dramatically over time, highly dependent on short term weather conditions). Two shallow sills, the Dr\o gden Sill (7 m deep, values in brackets refer to the local depth in the following) and the Darss Sill (18 m) form the passage to the first of several basins, the Arkona Basin (45 m), through which the North Sea water propagates as a deep gravity current. Through the Bornholm Channel, the Arkona Basin is connected to the Bornholm Basin (80 m), from where the bottom current flows over the S\l upsk Sill (60 m) and through the approximately 80 km long S\l upsk Furrow (90 m) southeast into the Gda\`{n}sk Basin (110 m) and northeast into the Eastern Gotland Basin (250 m). From there, the deep current can enter the easterly Gulf of Riga or the Western Gotland Basin, wherein the Landsort Deep (with 490 m the deepest point in the Baltic Sea) is located. The Gulf of Finnland forms the eastern boundary of the Baltic Sea and north of the Gotland Basin, the \r{A}land Sea separates the Baltic Proper in the south from the Bothnian Sea and the Bay of Bothnia in the north \citep[][]{reissmann2009}. Salt water entering at the Darss Sill needs approximately 2--6 months to exchange the bottom water in the Gotland Basin \citep[][]{balticsea}. 

The bathymetry of the Baltic Basin together with climate conditions and the exclusive water exchange via the Danish Straits leads to several notable features in the hydrographic conditions: Dense bottom water from the North Sea creates a permanent strong halocline that separates bottom from surface water, preventing ventilation of the water masses in the deep basins. Oxygen that enters through exchange with the atmosphere is not mixed below the halocline, leaving entraining North Sea water the only oxygen source. Large parts of the Baltic Sea become anoxic after long periods without significant salt water inflows, with severe implications on geochemical processes and the ecosystem of both the lower water column and the sediment-water interface. 

Since the only source of saline water are inflows from the North Sea, a salinity gradient is present across the Baltic Sea: The bottom current is entrained with overlying brackish water along its pathway (for example, volume of the gravity current increases by 53 \% when passing through the Arkona Basin, \citep[see][]{reissmann2009}), and slow entrainment of salt into the overlying water mass forms a NE--SW surface salinity gradient. Surface salinity drops from around 18 psu in the Danish Straits to 8 psu in the Arkona Basin, down to 6--7 psu in the Eastern Gotland Basin and even below 3 psu in the Bay of Bothnia and the Gulf of Finnland \citep[][]{balticsea}.

\subsubsection{Major Baltic Inflow December 2014}

Saline water can enter the Baltic Sea in two different ways. Baroclinic inflows are triggered by the horizontal salinity gradient between North and Baltic Sea and occur at calm wind conditions, mostly in summer \citep[][]{reissmann2009}. The main impact on the ventilation of the deep sea is, however, subject to the barotropic inflow events, called Major Baltic Inflows (MBIs). Those MBIs occur irregularly and are caused by special meteorological conditions: Firstly, high air pressure over the Baltic region and strong winds in westerly direction cause the mean sea level in the Baltic to drop by up to 50 cm, imposing a gradient in sea level elevation between Kattegat and Arkona Basin. Secondly, several weeks of wind in easterly direction with strong gales \citep[][]{balticsea, reissmann2009, mohrholz2015} push saline and oxygen rich water into the Baltic Sea.
These conditions occur primarily between October and February. Since records started in 1897, MBIs were on a rather regular base up to the 1970s. After this, long periods without inflow events decreased oxygen supply in the Baltic basins drastically. The last MBIs were in 1993 and 2003, but they interrupted the anoxic conditions in the deep water only for short periods of time \citep[][]{schinke1998, mohrholz2015}.

In November 2014, medium to strong easterly winds forced an outflow of Baltic sea water and consequently a drop in mean sea level of 57 cm. At the beginning of December, the wind changed to heavily westerly wind, pushing a strong barotropic inflow of saline water into the Baltic Sea. The main inflow period lasted until Christmas. During that time, an overall amount of 320 $\text{km}^3$ water was imported into the Baltic Sea, carrying approximately 4 Gt salt. The MBI in 2014 was the third largest inflow event recorded since 1880 and has the potential to ventilate the entire deep water of the Baltic. 

\subsection{German Coastal Seas and Sediment}

\section{Sediment Properties}

\subsection{Erosion, Transport Ways and Deposition}

\subsection{Sediment services}
