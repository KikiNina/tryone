\chapter{Introduction}
\label{kap-einleitung}

% Seitennummerierung auf arabische Zahlen umstellen, neu beginnend im Kapitel 1
\pagenumbering{arabic}

\chapterquote{Dämliches Zitat}{Zitierter Mensch}

\section{The Baltic Sea}

\subsection{History and Present State}

The Baltic Sea emerged from a subglacial lake at the end of the Weichselian Ice Age, approximately 12,000 BP and has since then gone through several stages of either fresh water conditions and salty or brackish states, when a connection to the open ocean was established. Indicators for this alternation were amongst others faunal changes, so the denomination of the evolution stages was determined by the abundant species of that time.

With the retreat of the glaciers at the end of the Weichselian Ice Age, the former subglacial lake together with rapidly melting glacial water formed the \textit{Baltic Ice Lake}. A landbridge between Sweden and the main land was created during a lowering of the Baltic Ice Lake 11,200 BP, the first connection since deglaciation. The sea level then was increased by glacial melt water until the lake began to drain over the Öresund sill. Presumably, a lowering of the water level by 25 to 28 m by drainage through a channel near Mt. Billingen in central Sweden sill took place very rapidly around 10,300 BP, when the retreating ice margin collapsed \citep[][]{bjoerk95,tikkanen2002}. 

This new stage of the Baltic with a lower sea level is known as \textit{Yoldia Sea} (from the marine mollusc \textit{Portlandia (Yolida) arctica} \citep[][]{schoning2001}). Immediately after the drainage, the Yoldia Sea was isolated from the ocean until global sea level rise and deglaciation formed a connection through the Närke Strait 300 years later \citep[][]{schoning2001}. This lead to brackish water conditions followed by decreasing salinity after the straits had become too narrow to allow water exchange another 100 to 300 years later \citep[][]{bjoerk95}.

After the glaciers had retreated, an uplift of the land took place that could not be compensated by global sea level rise. Connection to the ocean was closed again around 9,500 BP, the regime shifted to fresh water conditions and was now named \textit{Ancylus Lake} (after the freshwater gastropod \textit{Ancylus fluviatilis} \citep[][]{tikkanen2002}). Although narrow outlets existed, the surface of the Ancylus Lake rose up to 10 m above sea level, inundating extensive areas of land. This phase, called the Ancylus Transgression, ended around 9,200 BP, when the sea level height exceeded the height of the Darss Sill and water began to flow out through an evolving river system in the land connection between Sweden and Germany \citep[][]{tikkanen2002}. Around 9,000 BP this so-called Ancylus Regression ended when the Ancylus Lake surface height reached sea level. Drainage through the river system still took place, but no extensive saline intrusion happened for the next 1,000 years.

Ongoing global sea level rise deepened the connection between the Baltic Basin and the open ocean, letting more and more saline water intrude. A short transition stage called \textit{Mastogloia Sea} (\textit{Mastogloia} are diatoms that live in slightly saline conditions \citep[][]{eronen2001}) was followed by the \textit{Litorina Sea} (named after the brackish water gastropod \textit{Littorina littorea} \citep[][]{eronen2001}), when a great extent of saline water entered the Baltic basin through the Straits of Denmark in 7,500 BP \citep[][]{bjoerk95}. Transgression took place until the rise in ocean levels ended 6,000 to 5,000 BP. When the Straits of Denmark became shallower as land uplift still went on, water exchange was reduced and salinity in the Litorina Sea declined down to the present brackish level in 4,000 BP. Since then, it was named \textit{Limnea Sea} (from the snail \textit{Lymnaea ovata}), and became the present \textit{Mya Sea} (after the Northern American clam \textit{Mya arenaria}, which was brought into the Baltic by Viking ships \citep[][]{bjoerck2008}) 500 BP.

Today, the Balic Sea is connected to the North Sea via the Danish Straits: the \O resund and the Great and Little Belt together with the Fehmarn Belt. An exchange flow, where brackish water leaves the Baltic near the surface (annually around 947 $\text{km}^3$) and saline North Sea water enters at the bottom (nearly 500 $\text{km}^3$ a year on average, but frequency and strength of inflow events differ dramatically over time, highly dependent on short term weather conditions). Two shallow sills, the Dr\o gden Sill (7 m deep, values in brackets refer to the depth in the following) and the Darss Sill (18 m) form the passage to the first of several basins, the Arkona Basin (45 m), through which the North Sea water propagates as a deep gravity current. Through the Bornholm Channel, the Arkona Basin is connected to the Bornholm Basin (80 m), from where the bottom current flows over the S\l upsk Sill (60 m) and through the approximately 80 km long S\l upsk Furrow (90 m) southeast into the Gda\`{n}sk Basin (110 m) and northeast into the Eastern Gotland Basin (250 m). From there, water can enter the Gulf of Riga in the east or continue its way through the basins into the Western Gotland Basin, wherein the Landsort Deep (490 m, deepest point in the Baltic Sea) is located. Finally, the eastern boundary of the Baltic Sea is the Gulf of Finnland and north of the Gotland Basin, the \r{A}land Sea separates the Baltic Proper in the south from the Bothnian Sea and the Bay of Bothnia in the North \citep[][]{reissmann2009}. 3 MONATE ZITAT  The climate is humid, annual precipitation adds around 224 $\text{km}^3$ of water to the Baltic Sea, evaporation amounts to only 184 $\text{km}^3$. River discharge a total of 436 $\text{km}^3$ water per year. \citep[][]{reissmann2009}.

The bathymetry of the Baltic Basin together with climate conditions and the characteristic of the water exchange with the North Sea leads to several notable features in the hydrographic conditions. The dense bottom water from the North Sea creates a permanent halocline that separates the bottom from the surface water. 
entrainment
salinity gradient
anoxic, ventilation, oxygen
seasonal thermocline
mixing - slopes!

\subsubsection{Major Baltic Inflow 2014 - A Christmas Carol}

\subsection{German Coastal Seas and Sediment}

\section{Sediment Properties}

\subsection{Erosion, Transport Ways and Deposition}

\subsection{Sediment services}
