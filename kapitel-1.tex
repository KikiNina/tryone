\chapter{Einleitung}
\label{kap-einleitung}

% Seitennummerierung auf arabische Zahlen umstellen, neu beginnend im Kapitel 1
\pagenumbering{arabic}

\chapterquote{Ach, die Physik! Die ist ja für die Physiker viel zu schwer!}{David Hilbert (Bsp. für Einbindung eines einleitenden Zitats)}

Der Umfang der Bachelorarbeit sollte 20-30 Seiten betragen.
Die Gliederung und der Inhalt der Kapitel sind mit dem Betreuer der Arbeit abzusprechen. Maßgebend für inhaltliche und formale Anforderungen sind natürlich die (Bachelor, Master)-\-Prüfungs\-ordnungen.

\paragraph{Technische Hinweise:}
\begin{itemize}
  \item Grundlayout und Strukturierung dieser Vorlage basieren auf der Latex-Klasse ``hepthesis'' von Andy Buckley. Im Ordner ``anleitungen'' findet sich die Anleitung mit weiteren Optionen und Befehlen hierzu.
  \item Die Literaturliste orientiert sich an der DIN 1505. Dazu werden das Paket ``natbib'' mit erweiterten Zitationsbefehlen (s.a. Ordner ``anleitungen'') und die Stile (*.bst-Dateien) aus dem ``din1505''-Paket verwendet. 
  \item Das Dokument ist mit ``pdflatex'' zu übersetzen, also \textit{pdflatex abschlussarbeit} auf der Kommandozeile bzw. im Latex-Editor ``Kile'' mit der Projekt-Einstellung ``Schnell\-erstellung: PDF-Latex + ViewPDF''.
  \item Die Literaturliste im Dokument wird mit Hilfe der Literaturdatenbank (*.bib-Datei) erzeugt. ``Kile'' erledigt das automatisch. Der entsprechende Kommando\-zeilen-Befehl lautet \textit{bibtex abschlussarbeit}, danach muss noch einmal \textit{pdflatex abschlussarbeit} folgen. Dieser Vorgang muss -- wie bei Latex üblich -- eventuell mehrfach wiederholt werden.
  \item Umlaute und Sonderzeichen: Man muss darauf achten, dass die in der Latex-Präambel angegebene Code-Tabelle (\textit{usepackage inputenc}-Befehl) mit der im Latex-Editor verwendeten übereinstimmt. Neuere Linux-Distributionen verwenden automatisch Unicode (UTF8), damit werden im Prinzip alle bekannten Sonderzeichen und selbst nicht-lateinische Schriften wie kyrillisch, wie getippt angezeigt.
\end{itemize}

Hier folgt der Text zu Kapitel 1 mit Literaturverweisen wie \cite{Jackson1999,Einstein1925} oder
\cite{HaugKoch2004,Meinke2005,Holm1998}.

\section{Physikalische Grundlagen}
Der Hamiltonoperator eines Teilchens ist \cite{Nolting2007}
\begin{equation}\label{ham}
\hat{H}=\frac{\hat{\vec{p}\,}^2}{2m}\,+\,V(\hat{\vec{r}}\,)
\end{equation}
Der Erwartungswert der Energie ergibt sich aus 
\begin{align}
  \label{gl-erwartungswert}
  \Braket{\Psi_0|\hat H|\Psi_0} = E,
\end{align}
mit dem Hamiltonoperator aus Gl.\ \eqref{ham}.


\section{Besonderheiten des verwendeten Materials}

\subsection{Effektive-Massen-Näherung}
weitere Untergliederung m\"oglich
