\chapter{The Effect of Wind Waves on Sediment Properties}
\label{kap-waves}

Wind generated waves are a main contributor to the energetics in shallow water 
bodies. They generate surface mixed layers and influence the near bed dynamics 
even at considerable water depths. In shallow areas, wave induced maximal 
bottom stress mainly dominates over current induced turbulence 
\citep[][]{jonsson2004}. Consequently, waves play a major role in sediment 
transport and distribution in the Baltic Sea.

\section{Introduction}

DEFINITION PERMEABILITY. Sediments that are considered as permeable allow pore 
water flow and exchange by hydrodynamically imposed pressure differences. 
Biogeochemical processes in the sediment are strongly affected by this process, 
and potentially permeable sediments are present in estimately 70$\%$ of the 
global shelf seas. CITE EMERY1968

Permeability is related to grain size distribution and size and geometry of the 
interstices in the sediment. 



\section{Permeability Map}

\cite{forster2003} constructed a map reflecting the permeability in the western 
Baltic Sea using direct measurements of vertical permeability and estimates 
from grain size distribution and porosity. From 37 samples, hydraulic 
conductivity was measured and converted to permeability. From the same samples, 
grain size distribution was determined. An approximation for the permeability 
is the relation of Krumbein and Monk \citep[][]{krumbein1942}
\begin{equation}
 \label{kKM}
 k_{KM} = 7.5 \times 10^{-4} d_{50}^2 e^{-1.31 \sigma},
\end{equation}
where $d_{50}$ is the median diameter and $\sigma$ the standard deviation of 
the grain size distribution. This relation only holds for sufficiently well 
sorted sediments with $\sigma \leq 0.7$. This formula was found to overestimate 
the permeability, thus Eq.\ref{kKM} was divided by a correction factor of 2.6.  
With this new relation, permeability was estimated from two data sets of grain 
size distribution and interpolated. The data are shown in Fig.\ref{perm}.
\begin{figure}[ht]
 \includegraphics[width=16cm]{bilder/perm_k.png}
 \caption{Permeability data from \cite{forster2003}. Reprinted 
with permission. Pink line indicates the k=$2.5\times10^{-12}$ 
isoline.\label{perm}}
\end{figure}

\section{Wave model for the Baltic Sea}\label{balticswan}

The SWAN wave model version 41.01 was set up for the Baltic Sea including a 
nesting in the area of the German coastal sea. For the whole Baltic Sea, a grid 
resolution of $0.5^\circ \times 0.1^\circ $ in Latitude and Longitude was chosen 
in a domain reaching from $53.5^\circ \text{N } 9^\circ \text{E}$ to $66^\circ 
\text{N } 31^\circ \text{E}$. This includes the Kattegat, the passage from the 
Baltic to the North Sea, where boundary conditions were prescribed in the east: 
A constant JONSWAP spectrum with a wave height of $1$ m, a period of $5$ s and 
peak wave direction of $90^\circ$ with a high directional spreading. The passage 
from the North to the Baltic sea is very shallow, so the boundary conditions 
have negligible influence on the model outcome. Wave frequency range was set to 
0.1 to 1.8 Hz, which mirrors the actual wave frequencies in the Baltic 
Sea \citep[][]{balticsea}, with a resolution of $42$ frequency bins. 
Computations included a spin-up time of $10$ days, sufficient for wave modeling 
and were performed for the years 2013 and 2014. The model was forced by wind 
data from the German Weather Service (DWD), the computational time step was $1$ 
hour (note that the propagation scheme implemented in SWAN is implicit, so the 
time step is not restricted by numerical issues). 

The nesting area reached from $53.5^\circ \text{N } 10^\circ \text{E}$ to 
$55.5^\circ \text{N } 15^\circ \text{E}$, including the Western German coastal 
seas. Spatial resolution was $0.01^\circ \times 0.025^\circ $ in Latitude and 
Longitude and the computational time step was set to $15$ minutes. Boundary 
conditions came from the Baltic Sea model and all parameters were left 
unchanged.

In \fig{verify} the significant wave height calculated with the wave model was 
compared to data obtained with a wave rider buoy. The significant wave height 
is defined as the average wave height of the highest one-third of the waves. 
This value matches the visually observed wave height best and can be calculated 
from the zeroth-order moment if the variance density spectrum $m_0$ via 
$H_{sign} = 4 \sqrt{m_0}$.
\begin{figure}[ht]
 \includegraphics[width=9cm]{bilder/januar.pdf}
 \caption{Significant wave height calculated with the SWAN wave model (HR 
refers to the high resolved nested run) and measured wave height from the 
Arkona wave rider buoy.\label{verify}}
\end{figure}

In the context of a Master Thesis \citep[][]{masterarbeitronja}, the model 
setup for 2013 (still using the previous version SWAN 40.91AB) was 
extensively tested for numerical convergence and compared to measured data on 
several locations in the Baltic Sea. Wave parameters like significant wave 
height were in good agreement with obtained data, even in areas with complex 
topography. A brief description of wind wave properties and processes and a 
discussion the wave model SWAN can be found in Appendix B.

\subsection{Idealized Wind Situations}

Storm surges in the Baltic sea are caused by two large-scale wheater 
situations. Strong northwest or northeast winds generate the highest 
waves in the western Baltic \citep[][]{balticsea}. Northeast winds maximize the 
fetch and therefore the wave activity in German coastal waters. Strong 
northwesterly winds occur more frequently, but tend to last shorter. 

To model wave activity under the two scenarios described above, we forced the 
Baltic Sea wave model with constant wind from the northwest (NW) and the 
northeast (NE), respectively. Wind speed was set to 20 m~s$^{-1}$, which was 
the maximal wind speed reached during a strong northwest wind in January 2005.
We used the same resolution and configuration as described in 
section \ref{balticswan} and a the modeled time period sufficiently long to 
reach stationarity (10 days).

\section{Results}	

\section{Discussion}

\section{Conclusions}