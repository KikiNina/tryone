\chapter{Surface Gravity Waves}
\label{kap-waves}
The spectrum of vertical motions of the ocean surface consists of several types 
of waves, covering a broad frequency range.  Wind generated 
waves have periods around $0.25 - 30\,\text{s}$. Locally generated waves, called 
wind sea, are irregular and short crested. While propagating away from the 
generation area, they become long crested and more regular, their periods 
increase. These waves are called swell \citep[][]{holthuijsen2007}. 

Wind generated waves are a main contributor to the energetics in shallow water 
bodies. They generate surface mixed layers and influence the near bed dynamics 
even at considerable water depths. In shallow areas, wave induced maximal 
bottom stress mainly dominates over current induced turbulence 
\citep[][]{jonsson2004}. Consequently, waves play a 
mojor role in sediment transport and distribution in the Baltic Sea.

In this chapter, a brief description of wind wave properties und processes will 
be given, followed by a discussion of a state of the art wave model, which was 
set up for the Baltic Sea. Finally, the model outcome and its implications on 
the Baltic Sea sediments are analyzed.

\section{Evolution of waves}

\subsection{Wave Generation by Wind}

\subsection{Wave-wave Interaction}

\subsection{Dissipation of Energy}

\section{Sea State Description}
The most important method 
to describe a sea state is to treat ocean waves as a stochastic process to 
calculate a wave spectrum, from which parameters like wave height are derived. 
One possibility therefore is the random-phase/amplitude model, described in 
detail in \cite{holthuijsen2007}: The time series of surface elevation $ \eta 
(t)$ is assumed to be a superposition of $N$ sinusoidal waves, each with a 
distinct frequency $f_i$, amplitude $a_i$ and phase $\alpha_i$, written as
\begin{equation}
 \label{sinussum}
 \eta(t) = \sum_{i=1}^N a_i \cos (2\pi f_i t + \alpha_i).
\end{equation}
Considering $a_i$ and $\alpha_i$ to be independent random variables, 
probability density functions for both can be derived from observations. For 
waves in sufficiently deep water which are not too steep, it was found that the 
phase is uniformly distributed (and can therefore be omitted in the wave 
spectrum), while the amplitude is Rayleigh distributed with expected value 
$E\{a_i\} = \mu_i$ for each frequency $f_i$. The function $f_i \rightarrow 
E\{a_i\}$ is called the amplitude spectrum and already contains all 
(non-directional) information of the sea state. A more meaningful 
characterization is, however, the variance spectrum $f_i \rightarrow E\{a_i^2 
\slash 2\}$, as the variance is proportional to the wave energy contained in 
each frequency, with the proportionality constant $\rho g$. Here, $\rho$ denotes 
density of the water and $g$ the gravitational acceleration. By scaling the 
variance of each discrete frequeny with the frequency bin $\Delta f$ (i.e. the 
size of the frequency resolution) and formally assuming infinitely small 
frequency bins, the variance density spectrum is obtained. Multiplied by the 
factor $\rho g$, this is the energy density spectrum:
\begin{equation}
 \label{vardensspec}
 E(f) = \rho g \lim_{\Delta f \rightarrow 0} \frac{1}{\Delta f} E\{\frac{1}{2} 
a_i^2\}.
\end{equation}
For a harmonic wave, i.e. all wave energy is contained in one distinct 
frequency, $E(f)$ reduces to a delta function. For more irregular waves, the 
energy density spectrum broadens. 

The shape of this spectrum is not at all arbitrary, but was found to have 
rather uniform characteristics for some conditions. One of this 
characteristic shapes is the Pierson-Moskowitz spectrum. Regarding a developing 
sea state in deep water, it is known from observations, that the spectrum 
evolves from higher frequencies to low frequencies. From a dimensional argument 
(assuming that wave breaking limits the spectra on the high frequency side, and 
that this process is determined only by the gravitational acceleration and the 
wave frequency itself), the spectra for waves with a frequency higher than some 
peak frequency was found to be proportional to $E(f) \sim g^2f^{-5}$.

directional spectrum

\section{Wave modeling}

\subsection{Third Generation Wave Model SWAN}

The open source wave model SWAN, an acronym for \textbf{S}imulating \textbf{WA}ves \textbf{N}earshore, is a wave model optimized for applications in coastal waters. As the spatial resolution of model domains in coastal regions is often very fine to capture smaller features in topography, SWAN has an implicit propagation scheme implemented, assuring numerical stability even for timesteps, that violate the CFL criterion. 

The central equation that is solved for every timestep at every grid point is the action balance equation, that accounts not only for the energy distributions in the wave frequency space but also for frequency shifts by wave-current interaction. The action density $N=N(\sigma,\theta; x,y,t)$ is therefore dependend on the relative radian frequency $\sigma$, which is the shifted frequency in a system moving with the current:
\begin{equation}\label{ebe}
 \frac{\partial N}{\partial t} + \frac{\partial c_{g,x} N}{\partial x} + \frac{ \partial c_{g,y} N}{\partial y} + \frac{\partial c_{\theta} N}{\partial t} + \frac{\partial c_{\sigma} N}{\partial \sigma}= \frac{S_{in} + S_{nl} + S_{wc}}{\sigma},
\end{equation}
where the source terms on the right hand side refers to energy input by wind, nonlinear wave -- wave interaction and energy dissipation by whitecapping. $t, x ,y$ are time and spatial directions, $\theta$ is the direction of wave energy propagation and $c_{g,x}, c_{g,y}$ are the group speed in each spatial direction. The fifth term on the left had side represents the frequency shift by ambient currents.

In the absence of currents, this action balance equation reduces to the energy balance equation.
\begin{equation}\label{ebe}
 \frac{\partial E}{\partial t} + \frac{\partial c_{g,x} E}{\partial x} + \frac{ \partial c_{g,y} E}{\partial y} + \frac{\partial c_{\theta} E}{\partial t} = S_{in} + S_{nl} + S_{wc}.
\end{equation}
All equations in the following will be noted, consistent with the action balance equation, in terms of the relative radian frequency $\sigma$. Although various formulations for the different mechanisms below are implemented in SWAN, only the ones used in the setup for the Baltic Sea in chapter \ref{balticswan} are described.

\subsubsection{Wave generation by wind}

Wind data is provided to the model as the wind speed and direction at 10 m elevation, $U_{10}$. This value is converted to a friction velocity via
\begin{equation}
 u_\ast^2 = C_D U_{10}^2.
\end{equation}
Before version 41.01 of SWAN, the wind drag coefficient was linearly dependent on the wind speed with an imposed lower limit. This formulation was found to overestimate $C_D$ for strong winds. Therefore, \cite{zijlema2012} fitted a second-order polynomial to nearly 5000 observed wind drag coefficients in dependence of the wind speed and came up with 
\begin{equation}
 C_D = \left( 0.55 + 2.97 \dfrac{U_{10}}{U_{ref}} - 1.49 \left( \dfrac{U_{10}}{U_{ref}} \right)^2 \right) \times 10^{-3} ,
\end{equation}
where $U_{ref} = 31.5 \, \text{m s}^{-1}$ is the wind speed at which $C_D$ is maximal.

The source term for energy input by wind includes two mechanisms:
\begin{equation}\label{gen}
 S_{in} (f, \theta) = \alpha + \beta E(f,\theta),
\end{equation}
where $\alpha$ is the initial wave growth, parameterized with the following empirical expression by Cavaleri and Malanotte--Rizzoli
\begin{equation}
 \alpha = 
 \begin{cases}
  \dfrac{1.5 \cdot 10^{-3}}{g^2 2 \pi} \left(u_\ast \cos{(\theta - \theta_{wind})}\right)^4 G & \text{for } |\theta - \theta_{wind}| \le 90^\circ  \\
  0 & \text{for } |\theta - \theta_{wind}| \textgreater 90^\circ  \\
 \end{cases}
\end{equation}
where $G$ is a cut-off function to avoid wave growth at frequencies below the Pierson--Moskowitz frequency, $g$ the gravitational acceleration and $\theta_{wind}$ is the wind direction. 

Another implemented approach to quantify the initial wave growth is to impose the JONSWAP spectrum for young sea states. The non-dimensional peak wave frequency $\tilde{f}_{peak} = f_{peak} U_{10} / g$ is thereby derived using the spatial step size at each point as fetch $F$ with an empirical equation by \cite{kahma1992} for short fetch lengths:
\begin{equation}
 \tilde{f}_{peak} = 2.18 \tilde{F}^{-0.27},
\end{equation}
where $\tilde{F} = g F / U_{10}^2$. All other coefficients in the JONSWAP energy spectrum equation are derived from this peak frequency \citep[][]{holthuijsen2007}, and  wave directions are assumed to be $\cos^2$ distributed.

For the exponential wave growth term $\beta$ in \eqref{gen} the classical expression of \citep[][]{komen1984} is used:
\begin{equation}
 \beta = \max \{ 0,0.25 \frac{\rho_{air}}{\rho_{water}} \left[28 \frac{u_\ast}{c} \cos(\theta - \theta_{wind}) -1 \right] \} \sigma,
\end{equation}
with the phase velocity $c$ and $\rho_{air}$ and $\rho_{water}$ the densities of air and water.

\subsubsection{Nonlinear wave -- wave interaction}

Quadruplet wave -- wave interaction is calculated with the discrete--interaction approximation (DIA) by \citep[][]{hasselmann1985}. Only two constellations of wave quadruplets are considered to interact. The frequencies must fulfill
\begin{align*}
 \sigma_1 &= \sigma_2 = \sigma \\
 \sigma_3 &= 1.25 \sigma = \sigma^+ \\
 \sigma_4 &= 0.75 \sigma = \sigma^- .\\
\end{align*}
The wave directions must be equal for the waves with frequency $\sigma$, while the waves with frequencies $\sigma_3$ and $\sigma_4$ lie at angles of $\theta_1 = -11.5^\circ \; \left[ \text{or } \theta_1 = 11.5^\circ \right]$ and $\theta_2 = 33.6^\circ \; \left[ \text{or } \theta_2 = -33.6^\circ \right]$  to them. The contribution for each quadruplet to the source term $S_{nl}$ in Eq. (\ref{ebe}) is:
\begin{equation}
 S_{nl4}^\ast (\sigma, \theta) = 2 \delta S_{nl4} (\alpha_1 \sigma, \theta) - \delta S_{nl4} (\alpha_2 \sigma, \theta) - \delta S_{nl4} (\alpha_3 \sigma, \theta),
\end{equation}
where for $i \in \{1,2,3\}$
\begin{align*}
 \delta S_{nl4} ( \alpha_i \sigma, \theta ) = & 2.12 \cdot 10^{(-4)} \sigma^{11} \cdot [ ( 0.41 E(\alpha_i \sigma^+, \theta)E^2(\alpha_i \sigma, \theta) \\
 &+ 3.16 E(\alpha_i \sigma^-, \theta) ) - 54.6 E(\alpha_i \sigma, \theta) E(\alpha_i \sigma^+, \theta) E^2(\alpha_i \sigma^-, \theta) ] .
\end{align*}
The coefficients are $\alpha_1 = 1, \, \alpha_2 = 1.25, \, \alpha_3 = 0.75$. 

The above holds for infinite water depth and must be muliplied with a scaling factor 
\begin{equation}
 R ( \tilde{k}d ) = \max \{ 1+ \frac{22}{3}\tilde{k}d \left( 1-\frac{6}{7} \tilde{k}d \right) \times \exp(-1.25 \tilde{k}d) , 4.43 \} ,
\end{equation}
depending on the mean wave number $\tilde{k}$ and the the water depth $d$ to obtain the finite depth contribution to the source term for nonlinear wave -- wave interaction.

\subsubsection{Dissipation}

Wave energy is dissipated by white -- capping. The representation of this process originates from \citep[][]{hasselmann1974}:
\begin{equation}
 S_{wc} (\sigma, \theta ) = - 2.36 \cdot 10^{-5} \frac{k}{\tilde{k}} \left( \frac{\tilde{s}}{\tilde{s}_{PM}} \right)^4 \frac{\tilde{\sigma}}{\tilde{k}} k E(\sigma, \theta),
\end{equation}
with $\tilde{s} = \tilde{k} \sqrt{m_0}$ being the overall wave steepness, $k$ the wave number, $\tilde{\sigma}$ the mean wave frequency and $\tilde{s}_{PM} = \sqrt{3.02 \times 10^{-3}}$ the overall wave steepness of the Pierson -- Moskowitz spectrum.


\subsubsection{Wave model setup for the Baltic Sea}\label{balticswan}

The SWAN wave model version 41.01 was set up for the Baltic Sea including a nesting in the area of the German coastal sea. For the whole Baltic Sea, a grid resolution of $0.5^\circ \times 0.1^\circ $ in Latitude and Longitude was chosen in a domain reaching from $53.5^\circ \text{N } 9^\circ \text{E}$ to $66^\circ \text{N } 31^\circ \text{E}$. This includes the Kattegat, the passage from the Baltic to the North Sea, where boundary conditions were prescribed in the east: A constant JONSWAP spectrum with a wave height of $1$ m, a period of $5$ s and peak wave direction of $90^\circ$ with a high directional spreading. The passage from the North to the Baltic sea is very shallow, so the boundary conditions have negligible influence on the model outcome. Wave frequency range was set to $0.1 to 1.8 \text{ Hz}$, which mirrors the actual wave frequencies in the Baltic Sea \citep[][]{balticsea}, with a resolution of $42$ frequency bins. Computations included a spin-up time of $10$ days, sufficient for wave modeling and were performed for the years 2013 and 2014. The model was forced by wind data from the German Weather Service (DWD), the computational time step was $1$ hour (note that the propagation scheme implemented in SWAN is implicit, so the time step is not restricted by numerical issues). 

The nesting area reached from $53.5^\circ \text{N } 10^\circ \text{E}$ to $55.5^\circ \text{N } 15^\circ \text{E}$, including the Western German coastal seas. Spatial resolution was $0.01^\circ \times 0.025^\circ $ in Latitude and Longitude and the computational time step was set to $15$ minutes. Boundary conditions came from the Baltic Sea model and all parameters were left unchanged.

In the context of a Master Thesis \citep[][]{masterarbeitronja}, the model setup for 2013 was tested for numerical convergence and compared to measured data on several locations in the Baltic Sea. Wave parameters like significant wave height were in good agreement with obtained data, even in areas with complex topography.

\begin{figure}[h]
% \includegraphics[width=7cm]{bilder/arkona.pdf}
 \caption{Significant wave height calculated with the SWAN wave model and measured wave height from the Arkona wave rider buoy.}
\end{figure}

\section{Wave climate of the Baltic Sea}	

\subsection{Observations and Model Outcome}

\subsection{Implications on Sediment Transport}