% ``klassisches'', vollstaendig in Latex gesetztes Titelblatt
% Klassische Titelseite

\thispagestyle{empty}
\begin{center}
\vspace*{-2cm}
\includegraphics[width=0.75\textwidth]{unilogo-siegel-farbe}\\
\vspace*{3cm}
    {\titlefont \huge \onehalfspacing
	\thetitle 
    \par}%
   \vfill
%  \vskip 6em
    {\normalfont\normalcolor\bfseries
	\large
	Dissertation \\
	\large
%	im Studiengang Physik\\
	zur\\
	Erlangung des akademischen Grades\\
	doctor rerum naturalium (Dr. rer. nat.)\\ 
	der Mathematisch-Naturwissenschaftlichen Fakultät\\
	der Universität Rostock
    \par}%
\end{center}\par
%\vfill
\vspace*{2.5cm}
\noindent\begin{minipage}[b]{\textwidth}
{\bf
  \noindent vorgelegt von\\
  Kirstin Schulz, geb.~am 10.~Januar 1988 in 
Clausthal-Zellerfeld\\

%%% Nur für Pflichexemplare

%  \begin{tabbing}
%  Betreuer und 1.~Pr\"ufer :  \= PD Dr. Lars Umlauf, Leibniz Institut für 
%Ostseeforschung\\
% 2.~Pr\"ufer: \> ???, Universität Rostock \\
%  \end{tabbing}

  \noindent Rostock, den  1.~April 2010
  }
\end{minipage}




% Alternative: Titelblatt im Uni-Corporate-Design aus Word/OpenOffice-Vorlage erstellen,
% als PDF abspeichern/exportieren und hier mit pdfpages einbinden.
% \includepdf[pages={1}]{titelseite-cd.pdf}


\begin{abstract}[Abstract]
\thispagestyle{plain}
Sediment transport is governed by the complex interplay of many different 
processes. Near sloping topography, sediment transport can be triggered by 
asymmetries in the density stratification during an oscillatory current going 
up and down the slope. This process was investigated in detail by means of an 
idealized numerical model. It was found that the process depends on the 
strength of the vertical stratification and the slope angle, and residual 
sediment transport is directed upslope for most of the parameter constellations. 
A more profound numerical model was used to successfully reproduce observations 
of this process from the East China Sea, and to investigate the effect of Earth 
rotation. Another data set was analyzed to investigate the transport of sediment 
across the rim of a deep basin in the Baltic Sea. It was found that parts of 
the sediment transported there are dispersed across the bottom slope into 
shallower areas.
\end{abstract}


\begin{abstract}[Zusammenfassung]
\thispagestyle{plain}
\sloppy
Der Transport von Sedimenten wird von dem komplexen Zusammenspiel vieler 
verschiedener Prozesse bestimmt. Nahe eines geneigten Meeresbodens können 
Asymmetrien in der vertikalen Dichteschichtung während der Phasen einer 
oszillierenden Strömung quer zum Hang einen residuellen Sedimenttransport 
verursachen. Dieser Prozess wurde mit der Hilfe eines idealisierten numerischen 
Modells detailliert untersucht. Es hat sich gezeigt, dass die Stärke der 
vertikalen Dichteschichtung und der Winkel des Hanges eien großen Einfluss auf 
diesen Prozess haben, der residuelle Sedimenttransport aber für den Großteil 
der Parameterkombinationen hangaufwärts gerichtet ist. Das Model wurde 
erweitert, um erfolgreich Beobachtungen dieses Prozesses aus dem 
Ostchinesischen Meer zu reproduzieren und um den Einfluss der Erdrotation zu 
untersuchen. Ein weiteres Datenset wurde analysiert um den Sedimenttransport 
am Hang eines tiefen Beckens in der Ostsee zu untersuchen. Es wurde 
herausgefunden, das Sediment, welches einmal in das Becken hinein transportiert 
wurde, auch wieder in Richtung flacherer Gebiete verteilt werden kann.
\fussy
\end{abstract}
%\cleardoublepage%

\newpage
\thispagestyle{plain}
\begin{center}
 \textbf{Acknowledgements}
\end{center}

I like to express my deep gratitude to the people who surrounded and supported 
me during my PhD time. In the last three years I did not only get to know a 
totally new field of scientific work, but I met unbelievably wonderful 
people that made my time in Rostock to be a great experience. 

In first place, I thank my supervisor Lars Umlauf, who invested so much time 
and energy to make a physicist out of me. Thank you for encouraging me all the 
time and thank you for all your advice, and for helping me through the hard 
times when nothing went right.

I also thank the other three members of my thesis committee: Hans Burchard, 
thank you for introducing me to all the people on the conferences and for 
always offering your help. You were like a second supervisor for me. Henk 
Schuttelaars, thank you for all the wonderful conversations, in Havard as well 
as on the Christmas Market, and your advice on my work and my future. And 
finally Gregor Rehder, who was the most sympathetic chief scientist I ever went 
on a cruise with. 

I greatly thank my working group and the colleagues from the project. Thanks 
to all the people that shared some time in room 214 with me: Mahdi 
Mohammadi-Aragh, who helped me so much during my first time and provided me with 
the most important schedule ever, Kaveh Purkiani and Johannes Becherer, Merten 
Siegfried, Selina Müller and Madline Kniebusch, who all contributed to the 
familiar and relaxed atmosphere in our office. Special thanks goes to Chris 
Lappe, who untiringly answered all my questions about oceanography and who tried 
to explain internal waves to me at least a hundred times. Thanks to GETMan Knut 
Klingbeil, for permanent support with everything related to computers, and for 
your ability to encourage me. Thanks to Elisabeth Schulz, who I always looked up 
to. Thanks to Peter Holtermann for showing me Galvano, to Ulf Gräwe, 
Xaver Lange and Berkay Basdurak for providing support and a wonderful and 
relaxed working atmosphere. Thanks to Berit Recklebe for being the shield to 
protect me from administration and for being a friend from my first working day 
on. I also want to thank Toralf Heene and Sebastian Beier for the technical 
support and their patience with me, and of course the crew members of the 
research vessels Alkor, Elisabeth Mann Borgese and Maria S. Merian. Thanks to 
Claudia Morys, Dennis Bunke, Jana Wölfel, Mayya Gogina, Florian Cordes and Julia 
Regnery, who made the ships cruises an unforgettable experience and became my 
closest friends during my time in Rostock.\newpage
\thispagestyle{plain} I am so grateful that I had the opportunity to meet all of 
you and I hope we will stay in touch.

I also want to express my gratitude to Prof. Dr. Margit Rösler, who was my 
mentor during my years at the Technical University of Clausthal-Zellerfeld, and 
who taught me to never surrender to a problem. I would also like to thank 
Wiebke Klünder and Markus Schreiber for always being there for me, during study 
times and beyond.

In the end, I want to thank my perfect flatmate Josephina Seltenheim, for 
providing me shelter when I came to Rostock and for being such a wonderful 
person, and all my other friends in Berlin. I always loved to have you around. 
I want to thank my best friend Jennifer Smoch for your endless support with 
anything and the wonderful times we have together and my partner Christian 
Meier for his patience and for moving to the Netherlands with me. I thank my 
family, my grandmother Eva Eggeling and especially my mother and friend Maike, 
for creating a wonderful childhood for me and for supporting me in every 
possible way. Everything I have achieved so far I owe to you. I thank my amazing 
brother Paul for all the years we have been together and my father Uwe for 
everything he did for me. Finally, I particularly want to thank Marko 
Lipka, who shared his apartment, Socke the dog and his small family with me for 
the last two years. You and Klara were such a huge and wonderful part of my 
life here and I will miss you so much. Goodbye und Guten Flug. 

\textbf{Regarding the publication in chapter 2:} This study was carried out in 
the context of the
interdisciplinary project SECOS, funded by the German Federal Ministry
of Education and Research under grant 03F0666A, WP 2.1 (PI: Lars
Umlauf). We are grateful to Henk Schuttelaars (TU Delft), Hans
Burchard, Ulf Gr{\"a}we, and Knut Klingbeil (all IOW) for modeling
support and helpful comments on the manuscript. Computations were
carried out with a modified version of General Ocean Turbulence Model
(GOTM, www.gotm.net) with the SPM component implemented using the
Framework for Aquatic Biogeochemical Models (FABM,
www.sf.net/projects/fabm).
\newpage

\tableofcontents
