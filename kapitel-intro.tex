\chapter{Introduction}
\label{kap-intro}

% Seitennummerierung auf arabische Zahlen umstellen, neu beginnend im Kapitel 1
\pagenumbering{arabic}

% \chapterquote{Chaos is found in greatest abundance wherever order is being 
% sought. It always defeats order, because it is better organized.}{Terry 
% Pratchett, Interesting Times: The Play}
% 
% Everything in the world's oceans, from the movement of water masses 
% by the gravitational force of moon and sun, to the smallest drops of rain water 
% falling on the sea surface, is resulting from, and simultaneously contributing 
% to, a complex interplay of physical processes on countless scales. Science has 
% tried to disentangle these processes for hundreds of years, and is still doing 
% so. Open questions are answered by chopping them into smaller and smaller 
% chunks, putting them in order and then connecting the loose ends again. Every 
% new discovery, as small as it may be, withdraws a bit from the infinite 
% reservoir of the unknown, that we call chaos. 

The state of marine sediments is closely linked to the conditions and 
processes prevailing in the overlying water column. Discharged material, 
nutrients as well as pollutants, can be stored and transformed in the 
sediment, and released into the water column again. Additionally, the sea floor 
provides a habitat for numerous species in complex and fragile ecosystems. 
Sediment resuspension, transport, and deposition highly influence the above, 
and sediment dynamics are therefore of great interest for the understanding of 
this complex system. An interplay of countless physical processes determines 
the spreading of sediments. In this thesis, a process for sediment transport in 
the vicinity of a sloping bottom is investigated. 

Motivated by the outcome of studies on estuarine dynamics controlling sediment 
transport processes, a basic process study on sediment transport, induced by an 
oscillatory current moving over sloping topography, was carried out. The 
prerequisite of the investigated sediment transport process is, besides the 
vicinity to a bottom slope, only a vertical density stratification of the water 
column. These conditions are prevailing in many regions throughout the ocean, 
for example on continental shelves or near underwater basins. The investigated 
sediment transport process might therefore play an important role for 
the sediment dynamics all over the ocean. It was found that sediment is 
transported up the slope under most of the naturally occurring conditions. This 
study was published in 2016 in the Journal of Physical Oceanography under the 
title "Residual transport of suspended material by tidal straining near sloping 
topography" and is found in chapter \ref{kap-slope} of this thesis. 

During the creation of the model study, observational data from the East China 
Sea, which gave evidence for the natural occurrence of the investigated 
process, was analyzed and is now published in \cite{Endohetal2016a}. A second 
model study was carried out, aiming on the reproduction of the observations. 
Therefore, the numerical model was extended to include the effects of Earth 
rotation, which had been neglected in the previous study. Even though the model 
domain is only one-dimensional, the important points of the observed process 
were reproduced and the natural occurrence of slope-induced tidal straining was 
confirmed. Another important aim of this study was the investigation of the 
influence of the Coriolis force on the process, which had been neglected 
previously. This study was submitted to the Journal of Geophysical Research in 
October 2016, and the submitted version is included as chapter \ref{kap-jgr} in 
this thesis.

Besides the two model studies, ship-based field campaigns in coastal areas of 
the Baltic Sea were performed in the context of this thesis. An extensive data 
set, including microstructure turbulence data, was obtained. Parts of these 
data, which were obtained in the transition zone from a shallow area near the 
coast to a deeper basin, have been analyzed to investigate the sediment 
dynamics in the vicinity of the basin slope. Evidence for processes that 
transport parts of the sediment from the deep parts of the basin back to the 
shore were found. In chapter \ref{kap-einleitung}, general information about 
the Baltic Sea, with focus on near-bottom processes important for sediment 
resuspension and transport, are summarized. The data are the discussed in 
chapter \ref{kap-measure}.