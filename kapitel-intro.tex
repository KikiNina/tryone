\chapter{Introduction}
\label{kap-intro}

% Seitennummerierung auf arabische Zahlen umstellen, neu beginnend im Kapitel 1
\pagenumbering{arabic}

\chapterquote{Chaos is found in greatest abundance wherever order is being 
sought. It always defeats order, because it is better organized.}{Terry 
Pratchett, Interesting Times: The Play}

Everything in the world's oceans, from the movement of water masses 
by the gravitational force of moon and sun, to the smallest drops of rain water 
falling on the sea surface, is resulting from, and simultaneously contributing 
to, a complex interplay of physical processes on countless scales. Science has 
tried to disentangle these processes for hundreds of years, and is still doing 
so. Open questions are answered by chopping them into smaller and smaller 
chunks, putting them in order and then connecting the loose ends again. Every 
new discovery, as small as it may be, withdraws a bit from the infinite 
reservoir of the unknown, that we call chaos. 

In this thesis, a process for sediment transport in the vicinity of a 
sloping bottom is investigated. The sea floor provides a habitat for numerous 
species in complex and fragile ecosystems. The distribution of nutrients and 
pollutants attached to particles highly affects these ecosystems. The 
examination of the contributing physical processes is therefore an important 
step to improve our understanding of the interplay of physical and 
biogeochemical processes near the sea bed.

The prerequisites of the sediment transport process investigated in this thesis 
are, besides the vicinity to a bottom slope, a vertical density stratification 
of the water column and some kind of oscillatory current. These conditions do 
naturally occur throughout the oceans, for example on continental shelves or 
near underwater basins. The investigated process might therefore play an 
important role for sediment dynamics in many areas of the ocean.

In the following chapter 2, this process is explained and analyzed. With a 
highly idealized numerical model, the governing parameters and their 
influence on the process itself are determined. This part is published in 
the Journal of Physical Oceanography \citep[][]{schulzumlauf2016}. After that, 
in chapter 3, observational evidence of this process from the East China Sea 
are confirmed with an extended version of this idealized model. This part is my 
second publication and submitted to the Journal of Geophysical 
Research. Finally, a data set from the Baltic Sea obtained in the context of 
this thesis is analyzed with the aim of finding clues for the occurrence of the 
investigated process. Therefore, general information about the Baltic Sea are 
summarized in chapter 4, and the data set is discussed in chapter 5. 